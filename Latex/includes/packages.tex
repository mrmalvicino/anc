\usepackage[tmargin=17mm,bmargin=17mm,lmargin=27mm,rmargin=17mm]{geometry} % Formato de página

\usepackage[output-decimal-marker={,}]{siunitx} % Unidades del SI
    \sisetup{per-mode = fraction, fraction-function=\sfrac}

\usepackage{csquotes} % Usado por biblatex

\usepackage[
    backend = biber,
    style = apa,
    sortcites,
    url = true
    ]{biblatex} % Citas y referencias bibliográficas
    \addbibresource{includes/bibliography.bib} % Base de datos de referencias bibliográficas

\usepackage{graphicx}  % Define \includegraphics
    \graphicspath{{./images/}} % Define \graphicspath{{dir1}{dir2}} para incluir imágenes que estén en los directorios dir1 y dir2

\usepackage{multicol} % Separación en columnas
    \setlength\columnsep{18pt}

\usepackage[spanish]{babel}  % Traducciones y abreviaturas

\usepackage{amssymb}  % Símbolos y tipografía matemáticos

\usepackage{amsmath}  % Formato y estructura matemáticos

\usepackage[colorlinks=false, pdfborder={0 0 0}]{hyperref}  % Define \url{} para hipervínculos

\usepackage{fancyhdr}  % Encabezado y pie

\usepackage{pdfpages} % Define \includepdf

\usepackage{float} % Entorno para imágenes

\usepackage{enumitem} % Etiquetas para \begin{enumerate}
    \setlist[itemize]{itemsep=-1.5mm}

% \usepackage{fontspec} % Define \setmainfont{Times New Roman} (requiere LuaLatex)
%     \setmainfont{Times New Roman} % Documento con Times New Roman (requiere LuaLatex)

\usepackage{mathptmx}  % Documento con simil Times New Roman (compila con TexPDF)

\usepackage{caption} % Define \captionsetup{font=footnotesize}
    \captionsetup{font=small} % Ajusta el tamaño de la fuente de las leyendas